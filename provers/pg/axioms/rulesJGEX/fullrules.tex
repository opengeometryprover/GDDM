\documentclass[a4paper,11pt]{article}
\usepackage{a4wide}
\usepackage{color}

\begin{document}

\begin{center}
  \Large\bf Rules in JGEx
\end{center}

\bigskip

File \texttt{fullrules.txt}

\bigskip

\begin{enumerate}

\item \emph{The Definition of Full Angle.}

A full angle is defined as an ordered pair of two lines $u$ and $v$ denoted by $\angle[u,v]$.

\item $\angle[u,v] =  - \angle[v,u]$.

\item $\angle[u,v] = \angle[0]$, if and only if $u \| v$  (including $u = v$).

\item $\angle[u,v] = \angle[1]$,  if and only if $u\perp v$.

\item Addition of Full Angle.

  $\angle[u,s] + \angle[s,v] = \angle[u,v]$.  This rule to split an angle into two or more angles.

  For example, for any line $m$, we have  $\angle[u,v] =  \angle[u,m] +  \angle[m,v]$.

\item If $l \perp v$ or $\angle[l,v] = \angle[1]$, then
  $\angle[u,v] = \angle[u,l] + \angle[l,v] = \angle[u,l] + \angle[1]$.

  For example, if we have $AB \perp AD$, then  $\angle[BAC] = \angle[DAC] + \angle[1]$.

  \textcolor{red}{O exemplo está estranho}

\item The Isoceless Triangle Theorem for Full Angle.
  
 If $AC = BC$, we have $\angle[ABC] = \angle[CAB]$. Conversely, if
 $\angle[ABC] = \angle[CAB]$, then $AB = AC$ or $A, B$ and $C$ are collinear.

\item The Inscribed Angle Theorem.

  Points $A,B,C$ and $D$ are cylic iff $\angle[ABC] = \angle[ADC]$.

\item If Cyclic(A,B,M,N),  then $\angle[AB, AM] = \angle[BN,NM]$.  Hence we have  $\angle[AB, CD] = \angle[AB, AM] + \angle[AM, CD] = \angle[BN,NM]] + \angle[AM, CD]$.

\item Rule 11.

\item If $AB$ is the diameter of the circumcircle of triangle $ABC$
  and $D$ is on the circumcircle, then for any line $FG$, we have
  $\angle[CD,FG] = \angle[DBA] + \angle[ACB] + \angle[BC, FG]]$.
  
\item If $O$ is the intersection of Circle(O1, O P1 P3 I ) and Circle(O2, O P2 P4 J ) and Collinear(O, I, J), then $\angle[P1O, OP2] = \angle[P1P2, IP3] + \angle[JP4, P2P4]$.

\item 13
 If O is the circumcenter of triangle ABC, then \angle[BOC] = 2 * \angle[BAC].

\item 14
 For a circle(O, A B C D), if O is the intersection of AD and BC, then \angle[AOB] = 2 * \angle[CDA].

\item 15
 For a circle(O, A B), O is the center of circle, we have \angle[OAB] = \angle[ABO]. 

\item 16
 If triangle ABE is a right triangle and O is the midpoint of the hypotenuse(i.e. AB), then for any line CD, we have
 \angle[AB, CD] =  \angle[BAE] + \angle[1] + \angle[BE, CD]  

\item 17
 For a circle(O, ABC) and D is midpoint of BC, we have \angle[AB, EF] = \angle[BOD] + \angle[AC,EF]. 

\item 18
 For a circle(O,ABF) and E is the midpoint of AF, we have \angle[AB, CD] = \angle[EOA] + \angle[BF, CD].

\item 19
 If G is the orthocenter of A, B, C, \angle[AE, HI] = \angle[AE, BE] + \angle[BE, HI] = \angle[1] + \angle[BE, HI]

\item 20
If A is the incenter of P2, K, I,  \angle[AB, CD] = \angle[AB, BK] + \angle[BK, CD] = \angle[1] + \angle[KI, KA] + \angle[IA, IK] + \angle[BK, CD].

\item 21
 Equal Angle found in Geometry Deductive Database.

\item  22
 If $AB \perp AC$ and AB = AC, then \angle[AB,AC] = 2 * \angle[AB,BC].

\item  23
 If $AB \perp AC$ and AB = AC, then 2 * \angle[AB,DE] = 2 * \angle[AB,BC] + 2 * \angle[BC,DE] = \angle[1] +2 * \angle[BC,DE].

\item  24
 If OB // DE and A is on Circle(O,B), then 2 * \angle[AB, DE] = \angle[OA,OB]

\item  25
 A is on Circle(O,B), then 2 * \angle[AB, DE] = \angle[OA,OB] + 2 * \angle[OA,DE]

\item  26
 CD is the diameter of Circle(A, BCD), $AB \perp CD$, then  2 * \angle[DB,BA]  = \angle[0].

\item  27
 A, B, C, D, E are cyclic and AC = AB, then 2*\angle[AB, BC] =  \angle[CE, ED].

\item  28
 If A, B, E, F, G are cyclic and AE = AF, then 2 * \angle[AB, CD] =  2 * \angle[BE, CD] +  \angle[EG, GF].

\item  29
 If AB = AC = BC, i.e., triangle ABC is an equilateral triangle, then 3 * \angle[AB, BC] = \angle[0].  
\end{enumerate}

\end{document}

\end{enumerate}

\end{document}
%%% Local Variables:
%%% mode: latex
%%% TeX-master: t
%%% End:
